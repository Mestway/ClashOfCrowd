%!TEX root=./proposal.tex

\section{Related Work}

%What existing understanding of the problem has been developed?
%For a research proposal, this will briefly cover the most important related work in the space you are exploring.
%For a design proposal, this will introduce existing solutions, why they fall short, and the potential opportunity.

\paragraph{Interactive Machine Learning}
Interactive machine learning systems mainly try to either explain the corresponding models~\cite{ren2017squares, alsallakh2014visual}, or to accept feedback from users in order to improve or customize the model, or to compensate the lack of data.

The manipulations fall into two categories: 
\emph{labeling}, where users provide more training data to the model (e.g., ELUCIDEBUG~\cite{kulesza2015principles}, Visual Classifier Training~\cite{heimerl2012visual}), and \emph{feature engineering}, where users specify the included features or the feature weights based on their domain knowledge (e.g., INFUSE~\cite{krause2014infuse}, FeatureInsight~\cite{brooks2015featureinsight}).
More examples can be found in surveys~\cite{Brown2016HumanMachineLearnerIT}. 
Though Tam et al.~\cite{tam2017analysis} mentioned that when constructing a decision tree, compared to the machine, humans are more capable in looking ahead into the long-term effect of setting one node, understanding combined feature effects, and incorporating domain knowledge into their decision process.
However, little work has been done in understanding if such domain knowledge provided reflect people's over-confidence in themselves or how they are (not) aligned with the automatically learnt knowledge. 
This is the main focus of our work.

\paragraph{Competitive Human Computation Games}
